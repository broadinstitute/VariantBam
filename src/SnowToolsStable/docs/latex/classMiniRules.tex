\section{Mini\-Rules Class Reference}
\label{classMiniRules}\index{MiniRules@{MiniRules}}
{\tt \#include $<$Mini\-Rules.h$>$}

\subsection*{Public Member Functions}
\begin{CompactItemize}
\item 
bool \textbf{is\-Valid} (Read \&r)\label{classMiniRules_bb41521aecc3b9a06435b9ca30f13263}

\item 
void \textbf{set\-Region\-From\-File} (const string \&file)\label{classMiniRules_86dfcf6341a058f0a80cf7820c8bbb3b}

\item 
bool \textbf{is\-Read\-Overlapping\-Region} (Read \&r)\label{classMiniRules_630d29da8c7a301306f1453ca4edfd46}

\item 
size\_\-t \textbf{size} () const \label{classMiniRules_dbbcd299c7a5300ca5342628abd3ada1}

\end{CompactItemize}
\subsection*{Friends}
\begin{CompactItemize}
\item 
class \textbf{Mini\-Rules\-Collection}\label{classMiniRules_0096f479fba63a9279b4563f1035fbda}

\item 
ostream \& \textbf{operator$<$$<$} (ostream \&out, const \bf{Mini\-Rules} \&mr)\label{classMiniRules_b5b68bfffca996f88c023a51410e47fd}

\end{CompactItemize}


\subsection{Detailed Description}
Define a set of rules for creating a variant bam. The syntax is: all@!isize:[0,800],mapq:[0,60] region rule1:[0,800],mapq:[0,60] rule2@!isize[0,800]:mapq[0,60],:ardclip:supplementary:duplicate:qcfail rule3@

A file of NA indicates that the rule should be applied genome-wide. The ordering of the lines sets the hierarchical rule. For instance, a rule on line 2 will be applied before a rule on line 3 for all regions that are the union of regions in level 3 and below.

e.g. Level 3 region file has region chr1 100 1000 Level 2 region file has region chr1 150 1200 The union of these will produce a new region chr1 100 1200, with level 2 



The documentation for this class was generated from the following file:\begin{CompactItemize}
\item 
/xchip/gistic/Jeremiah/GIT/isva/Snow\-Tools/src/Snow\-Tools/Mini\-Rules.h\end{CompactItemize}
